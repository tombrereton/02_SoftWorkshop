\documentclass[12pt]{article}
\scrollmode
\thispagestyle{empty}
%\usepackage{latexsym}
\usepackage{fancyhdr}
\def\myfooter{\vfill{\footnotesize\noindent\copyright\ 2016, Manfred Kerber, School of Computer Science, University of Birmingham}}
%\usepackage{colordvi}
\usepackage{url}
%\usepackage{bbm}
%\usepackage{verbatim}
%%%%%%%%%
%\usepackage{listings}
%\lstloadlanguages{Java,csh}
%\lstset{basicstyle=\ttfamily\normalsize,columns=fixed,breaklines=true,language=Java,showstringspaces=false}
%\lstMakeShortInline|

%%%%%%%%%%

\newif\ifslide\slidetrue
\newif\ifnotes\notesfalse
\newif\ifColour\Colourfalse
\newif\ifpdf\pdffalse
\usepackage{code}
\usepackage{epsfig}
\usepackage{a4wide}
%\usepackage{pstricks,pst-node}
%\addtolength{\textheight}{2.5cm}%{4ex}
\newcommand{\myhead}[1]{\begin{center}\large\bf #1\end{center}}
%\input{ex-macros} 
%\includeonly{linclass}
\begin{document}
\myhead{Lab lecture exercises -- 11 November 2016}

\begin{enumerate}
\item Assume the \texttt{BankAccount} class from last Wednesday,
  follow the Canvas pages
  \url{https://canvas.bham.ac.uk/courses/21955} at Teaching
  \texttt{wk07.zip}.

 Create
  a sub-class \texttt{IslamicMortgageAccount} which has three additional
  fields:
\begin{itemize} 
\item \texttt{private int months}, which specifies the number of
  months over which the mortgage has to be paid back, 
\item \texttt{private long payout}, the amount paid out to the customer initially (to buy the house), and
\item \texttt{private long fee}, the upfront fees involved.
\end{itemize}
Note that for an Islamic mortgage account no interest is paid. All
fees for the mortgage are to be included in the initial
\texttt{fee}. The initial balance on the account is the negative of
the sum of the \texttt{payout} and the \texttt{fee}.

Implement for the class a constructor as well as getters and setters for the new
field variables. Furthermore write two methods \texttt{public int
  initialPayment()} and \texttt{public int followUpPayment()} such that
the payments are evenly distributed over
\texttt{month -1} followup months (given as
\texttt{followUpPayment()}), the remainder has to be paid in the first month
as \texttt{public int initialPayment()}. Also write an appropriate
\texttt{toString()} method which specifies in addition to the
  information given for any \texttt{BankAccount} information about the
  \texttt{payout}, the \texttt{fee}, the \texttt{months}, the
  \texttt{initialPayment} and the \texttt{followUpPayment}.

  It should not be possible to make any withdrawals or transfers of money from an
  \texttt{IslamicMortgageAccount}. How can we guarantee this?

For instance, assume a mortgage (payout plus fee) of pounds 100000
over 240 months, then this would be paid back as 239 equal payments of
pounds 418 plus pounds 98 in the first month. That is,
\texttt{initialPayment()} should return 98 and
\texttt{followUpPayment()} should return 418.

\vfill
\hfill p.t.o.
\newpage

\item In this exercise you should extend the above mentioned \texttt{BankAccount} class to create a \texttt{SavingsAccount} which has two additional fields:
\begin{itemize} 
\item \texttt{private double interestRate}, which specifies the annual interest rate that is paid for the money in the account.
\item \texttt{private ArrayList<Transaction> transactions}, an ArrayList that contains all transactions (initial balance, deposit, or withdrawal).
\end{itemize}

In addition to the usual constructor, getters and setters, and
\texttt{toString()} method, you should write a method \texttt{public
  int annualInterest()} which computes the total interest accumulated
over the year. Note that the interest is paid pro rata for each of the 365
days of the year (if the year has 366 days, no interest is paid for
the 366th day). 

For example, if the interest rate is 1\% and for the first 200 days
there are $\pounds 1000$ in the account, for this period an interest
of $\pounds 1000 * 0.01 * {200\over 365}$, i.e., $\pounds 5.47$,
accrues. 

If on the first day after that period another $\pounds 1000$ are
paid in, for the remaining 165 days, the interest is computed on the
$\pounds 2000$ that are in the account, that is, $\pounds 2000
* 0.01 * {165\over 365}$. I.e., $\pounds 9.04$ are accrued for this
period. This gives a total annual interest of $\pounds 5.47 + \pounds
9.04$, which is rounded to $\pounds 15$.

You have also to implement a \texttt{Transaction} class, where a single transaction should store four pieces of information:
\begin{itemize}
\item the day of the year (1--365), when the transaction took place (there is no interest paid for the 366th day in a leap year),
\item the type of transaction (initial balance, deposit, or withdrawal),
\item the amount of the transaction, and
\item the balance after the transaction took place.
\end{itemize}
\end{enumerate}
Note that single days should NOT be broken down further to compute
e.g., an hourly interest. For computing the daily interest, the last
balance of that day should be taken.

\myfooter
\end{document}

%%% Local Variables: 
%%% mode: latex
%%% TeX-master: t
%%% End: 
