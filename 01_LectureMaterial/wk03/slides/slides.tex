\documentclass{beamer}
\def\java{\texttt{Java}}
\def\mytoday{12 October 2016}
\usepackage{url}
\usepackage{verbatim}
\usepackage{color}
\usepackage{listings}
\usepackage{code}
%\usepackage{bm}
\usepackage{beamerthemesplit}
\defbeamertemplate*{footline}{infolines theme}{
\hspace*{2ex} 
\copyright Manfred Kerber\hfill
   \insertpagenumber{} / \insertpresentationendpage \hspace*{2ex}
  \vskip1ex}
\usepackage{pgfpages}\def\mpause{\pause}
%\usepackage{pgfpages}\pgfpagesuselayout{8 on 1}[a4paper,border shrink=1mm, landscape]\def\mpause{}

\def\mcolor#1#2{\rule{0ex}{0ex}\color{#1}#2\color{black}{}}
\usetheme{Copenhagen}
%\setbeamercolor{title}{fg=red!80!black,bg=red!20!white}
\makeatletter % code block to allow custom labels to be cross-ref'ed; see comp.text.tex "customized display labels cross-ref'd"
\begin{document}

\title{MSc/ICY Software Workshop\\
Type Casting, Syntactic Sugar\\
Conditionals, Loops, Loop Invariants\\
Arrays, ArrayLists}


\author[Manfred~Kerber]{\begin{tabular}{ll}
\mcolor{blue}{Manfred Kerber} &   {\tt www.cs.bham.ac.uk/\~{}mmk}\\
\end{tabular}}

\date{\mytoday}

\begin{frame}
\titlepage


\end{frame}


\begin{frame}
\frametitle{Characters and Strings}
\texttt{String s;}\\
\texttt{s = "Hello, Java";}


With \mcolor{blue}{\texttt{s.length()}} you get the length of string \texttt{s}.

With \mcolor{blue}{\texttt{s.charAt(4)}} the 4th character in the string \texttt{s}.
(Careful, we start to count with zero!)

\mcolor{blue}{\texttt{s.substring(0,4)}} returns a substring of \texttt{s} from the 0th
element (inclusively) to the 4th element (exclusively).


\end{frame}

\begin{frame}
\frametitle{Type Casting}

Some types can be converted, some not. Examples are:
\texttt{byte b;}\\
\texttt{short s;}\\
\texttt{int i;}\\
\texttt{long l;}\\[3ex]
\texttt{float f;}\\
\texttt{double d;}\\[3ex]
\texttt{char c;}\\
\texttt{l = Long.MAX\_VALUE / 48;}\\
\texttt{i = \mcolor{blue}{(int)} l;}\\
\texttt{s = \mcolor{blue}{(short)} i;}\\
\texttt{b = \mcolor{blue}{(byte)} s;}\\
\texttt{f = \mcolor{blue}{(float)} i;}\\
\end{frame}

\begin{frame}
\frametitle{Type Casting (Cont'd)}
\texttt{d = 123.856778;}\\
\texttt{i = \mcolor{blue}{(int)} d;}\\[3ex]
\texttt{i = 123;}\\
\texttt{c = \mcolor{blue}{(char)} i;}\\[3ex]
\texttt{c = 'A';}\\
\texttt{i = \mcolor{blue}{(int)} c;}\\
\end{frame}

\begin{frame}
\frametitle{Some syntactic sugar}
\verbatiminput{sugar1.java}
\end{frame}

\begin{frame}
\frametitle{Some syntactic sugar (Cont'd)}
\verbatiminput{sugar2.java}
\end{frame}

\begin{frame}
\frametitle{\texttt{static}, \texttt{final}, \texttt{public}, \texttt{private}}
When defining a class we typically initialize objects with variables,
e.g. when we define a \texttt{BankAccount} we may have an account
number and an initial balance. Since the account number never
changes, we can enforce this by declaring:

\texttt{\mcolor{blue}{final int} ACCOUNT\_NUMBER = acNumber;}

Obviously the account numbers should be \mcolor{red}{different} for different
accounts.
\end{frame}

\begin{frame}
  \frametitle{\texttt{static}, \texttt{final}, \texttt{public}, \texttt{private} (Cont'd)}
In contrast when we speak about variables which are the same for
all objects in a class -- e.g., the interest rate could be the same for
all accounts of a particular type -- we call these variables \texttt{\mcolor{blue}{static}}.

E.g., if we want to have a verbose and non-verbose mode for all
objects in a class \texttt{C} then we can define this by

\texttt{\mcolor{blue}{public static boolean verbose;}}

and access it as

\texttt{C.verbose = false;}
\end{frame}

\begin{frame}
  \frametitle{\texttt{static}, \texttt{final}, \texttt{public}, \texttt{private} (Cont'd)}
If something is \mcolor{blue}{\texttt{static}} \textit{and} \mcolor{blue}{\texttt{final}} we can declare this by

\texttt{\mcolor{blue}{static final double CM\_PER\_INCH = 2.54;}}

If \mcolor{blue}{\texttt{public}} or \mcolor{blue}{\texttt{private}} in addition:

\texttt{\mcolor{blue}{public static final double CM\_PER\_INCH = 2.54;}}

or

\texttt{\mcolor{blue}{private static final double CM\_PER\_INCH = 2.54;}}
\end{frame}



\begin{frame}
\frametitle{Conditionals -- \texttt{if}}

\mcolor{blue}{Remember BankAccount: We want to allow withdrawals only
maximally until the balance is zero. How?}\mpause

\mcolor{red}{\textsl{If the condition is \texttt{true} the body is evaluated}}

\texttt{\ \mcolor{blue}{if} (<cond>)\{ }\\
\texttt{\ \ \ \ <command>*}\\
\texttt{\ \}}\\[2ex]

In case of one \texttt{command} also

\texttt{\ \mcolor{blue}{if} (<cond>) }\\
\texttt{\ \ \ \ <command>}\\[3ex]

\mcolor{red}{Example:}\\[1ex]

\texttt{\ \mcolor{blue}{if} (x < 0)\{ }\\
\texttt{\ \ \ \ \ x = -x;}\\
\texttt{\ \}}\\
  
\end{frame}



\begin{frame}
\frametitle{Conditionals -- \texttt{if else}}
\mcolor{red}{\textsl{If the condition is \texttt{true} the `then' part
is evaluated else the `else' part}}

\renewcommand{\baselinestretch}{0}
\texttt{\ \mcolor{blue}{if} (<cond>) \{}\\
\texttt{\ \ \ \ <command>*}\\
\texttt{\ \}\ \mcolor{blue}{else}\ \{}\\
\texttt{\ \ \ \ <command>*}\\
\texttt{\ \}}\\[3ex]

\mcolor{red}{Example:}\\[1ex]

\texttt{\ \mcolor{blue}{if} (x >= 0) \{}\\
\texttt{\ \ \ \ \ x = Math.sqrt(x);}\\
\texttt{\ \}\ \mcolor{blue}{else}\ \{}\\
\texttt{\ \ \ \ \ x = Math.sqrt(-x);}\\
\texttt{\ \}}\\
\end{frame}

\begin{frame}
\frametitle{Nested \texttt{if} statements}
\texttt{String str = "evening";}\\
\texttt{\mcolor{blue}{if} (str.equals("morning"))}\\
\texttt{\ \ System.out.println("Have a good day");}\\
\texttt{\mcolor{blue}{else} \mcolor{blue}{if} (str.equals("noon"))}\\
\texttt{\ \ System.out.println("Enjoy your lunch.");}\\
\texttt{\mcolor{blue}{else} \mcolor{blue}{if} (str.equals("afternoon"))}\\
\texttt{\ \ System.out.println("Good afternoon, see you soon");}\\
\texttt{\mcolor{blue}{else} \mcolor{blue}{if} (str.equals("evening"))}\\
\texttt{\ \ System.out.println("See you tomorrow");}\\
\texttt{\mcolor{blue}{else} }\\
\texttt{\ \ System.out.println("Oops. I don't understand.");}\\
\end{frame}

\begin{frame}
\frametitle{\texttt{switch} statements}

\mcolor{blue}{\texttt{switch}} provides a convenient way to select between different elements of
\texttt{byte}, \texttt{short}, \texttt{char}, and \texttt{int} data types with the syntax:\\[3ex]
\texttt{\ \ \mcolor{blue}{switch}(var)\{}\\
\texttt{\ \ \ \ \ \ \mcolor{blue}{case} value1\mcolor{blue}{:} ... \mcolor{blue}{break};}\\
\texttt{\ \ \ \ \ \ \mcolor{blue}{case} value2\mcolor{blue}{:} ... \mcolor{blue}{break};}\\
\texttt{\ \ \ \ \ \ ...}\\
\texttt{\ \ \ \ \ \ \mcolor{blue}{default}\mcolor{blue}{:} ... \mcolor{blue}{break};}\\
\texttt{\ \ \ \ \ \}}\\
\bigskip

If the default is missing and none of the cases occurs then the switch
statement does nothing.
\end{frame}

\begin{frame}
\frametitle{\texttt{while} loop}

\mcolor{red}{In a} \mcolor{blue}{\texttt{while}} \mcolor{red}{loop we distinguish the condition
(included in round brackets) and the body of the loop (included in
curly brackets). If the condition is true the body is executed
repeatedly until the condition is false after executing the body in
full.}

Syntax\bigskip

\texttt{\ \ \mcolor{blue}{while (}<cond>\mcolor{blue}{)} \{}\\
\texttt{\ \ <command>*}\\
\texttt{\ \ \}}\bigskip

Example:\bigskip

\texttt{int i = 0 ;}\\
\texttt{\mcolor{blue}{while (}i <  7\mcolor{blue}{)} \{}\\
\texttt{\ \ i = i+1; }\\
\texttt{\ \ System.out.print(i + " ");}\\
\texttt{\}}\\
\end{frame}

\begin{frame}
\frametitle{\texttt{for} loops}

\mcolor{red}{A} \mcolor{blue}{\texttt{for}} \mcolor{red}{loop is similar to a while loop, however, in the round brackets
we declare and initialize an iteration variable, separate by a semicolon the
termination condition and again by a semicolon the update expression for the
iteration variable.}

Syntax e.g.,\bigskip

\texttt{\ \mcolor{blue}{for (}int i = 0 \mcolor{blue}{;} i <=n \mcolor{blue}{;} i++\mcolor{blue}{)} \{}\\
\texttt{\ \ \ \ \ \  System.out.print(i + " ");}\\
\texttt{\ \ \ \ \ \}}\\
\end{frame}

\begin{frame}
\frametitle{A loop}
\verbatiminput{gcd.java}

\mcolor{blue}{\textbf{What does it do?}}\\
\mcolor{blue}{\textbf{Does it terminate?}}
\end{frame}


\begin{frame}
\frametitle{Loop Invariant}

\mcolor{blue}{Need good documentation of loops, in particular, a loop invariant
 in order to understand the loops}

\verbatiminput{gcd-inv.java}
\end{frame}


\begin{frame}
\frametitle{Arrays}
\mcolor{blue}{`An array is a data structure for storing a collection
  of values of the same type'} [Horstmann, Cornell, Core Java, p.90]

E.g., 
\verbatiminput{array.java}

Careful: The lowest index \texttt{i} is 0 the biggest \texttt{length-1},
that is, in this case \texttt{99}. 
\mcolor{blue}{It is easy to make mistakes involving array bounds. Hence, there should be test cases which check them.}
\end{frame}

\begin{frame}
\frametitle{Loops on Arrays}
There are different possibilities:\\[1ex]
\texttt{for (int i=0; i < a.length; i++)\{}\\
\texttt{\ \ \ \ \ System.out.print(a[i] + " ");}\\
\texttt{\}}\bigskip

Better: write\\[1ex]
\texttt{for (int element : a)\{}\\
\texttt{\ \ \ \ \ System.out.print(element + " ");}\\
\texttt{\}}
\end{frame}

\begin{frame}
\frametitle{Initialization of Arrays}

\texttt{\ \ int [] c = \{2, 5, 24, 6\};}\\\bigskip
  
\texttt{\ \ for (int element : c)\{}\\
\texttt{\ \ \ \ \ \ \ \ System.out.print(element + " ");}\\
\texttt{\ \ \}}\\
\end{frame}

\begin{frame}
\frametitle{Two-dimensional arrays}
Taken from [Deitel and Deitel, Java, 8th ed., 7.9, p. 274]

A two-dimensional array is an array of one-dimensional arrays.
\texttt{a[row][column]} declared and initialized e.g., by

\verbatiminput{two-dim-array-ex.java}

\end{frame}

\begin{frame}
\frametitle{Two-dimensional arrays (Cont'd)}
Example: TitTacToe taken from [Horstmann, Big Java, p. 288]

\verbatiminput{two-dim-array.java}

\end{frame}


\begin{frame}
\frametitle{Two-dimensional arrays (Cont'd)}
Example multiplication table:
\verbatiminput{multiplication.java}
\end{frame}

\begin{frame}
\frametitle{System.out.printf\\
Two-dimensional arrays (Cont'd)}
Two-dimensional arrays can be be initialized in an easy way
as shown in the example.
\verbatiminput{two-dim.java}
(see \texttt{PrintfTest.java})
\end{frame}
\begin{frame}
\frametitle{\texttt{ArrayList}}

Comparable to arrays, but without fixed size.\\
Only on objects. E.g.,

\def\baselinestretch{0.5}\def\arraystretch{0.5}
\begin{itemize}\def\arraystretch{0.5}
\item \texttt{ArrayList<String> items = new ArrayList<String>();}
\item \texttt{ArrayList<String> items = new ArrayList<String>(1000);}
\item \texttt{items.add("newString");}
\item \texttt{item.size()} corresponds to \texttt{length} of an array.
%\item \texttt{void ensureCapacity(int capacity);} minimal capacity
\item \texttt{void trimToSize()}; reduce storage size.
\end{itemize}
\end{frame}

\begin{frame}
\frametitle{\texttt{ArrayList} (Cont'd)}
  \texttt{ArrayList} is not a basic construct, a particular library
  has to be loaded. This is done by writing at the top of the class file

\mcolor{blue}{\texttt{import java.util.ArrayList;}}

More information can be found from the API (Application Programming
Interface).  The API documentation is part of the JDK (Java
Development Kit) from Oracle's Java pages.
\end{frame}

\end{document}
