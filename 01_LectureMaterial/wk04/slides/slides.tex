\documentclass{beamer}
\def\java{\texttt{Java}}
\def\mytoday{19 October 2016}
\usepackage{pgfpages}\def\mpause{\pause}
%\usepackage{pgfpages}\pgfpagesuselayout{8 on 1}[a4paper,border shrink=1mm, landscape]\def\mpause{}
\usepackage{url}
\usepackage{verbatim}
\usepackage{color}
\usepackage{listings}
\usepackage{code}
\usepackage{textcomp}

%\usepackage{bm}
\usepackage{beamerthemesplit}
\defbeamertemplate*{footline}{infolines theme}{
\hspace*{2ex}    \insertframenumber{} / \inserttotalframenumber\hspace*{2ex} 
\copyright Manfred Kerber
%   \insertpagenumber{} / \insertpresentationendpage \hspace*{2ex}
  \vskip1ex}

\def\mcolor#1#2{\rule{0ex}{0ex}\color{#1}#2\color{black}{}}
\usetheme{Copenhagen}
%\setbeamercolor{title}{fg=red!80!black,bg=red!20!white}\def\mpause{\pause}
\makeatletter % code block to allow custom labels to be cross-ref'ed; see comp.text.tex "customized display labels cross-ref'd" 

\begin{document}

\title{MSc/ICY Software Workshop\\
Exception Handling, Assertions\\Scanner, Patterns\\File Input/Output}

\author[Manfred~Kerber]{\begin{tabular}{ll}
\mcolor{blue}{Manfred Kerber} &   {\tt www.cs.bham.ac.uk/\~{}mmk}\\
\end{tabular}}

\date{\mytoday}

\begin{frame}
\titlepage
\end{frame}

\begin{frame}
\frametitle{Classes and Objects}

The information we have about a particular object is encapsulated in
so-called \mcolor{blue}{field variables}. First, we have to clarify
which ones that should be.

In order to create and manipulate objects we always have:


\begin{itemize}
\item At least one \mcolor{blue}{constructor} (for the creation of objects)
\item \mcolor{blue}{getter}s are methods to get the components of objects back.
\item \mcolor{blue}{setter}s are methods to change components of objects.
\item The \mcolor{blue}{\texttt{toString()}} method is used when the object is to be printed. Without it, an object is not printed in a human readable way.
\item In order to check two objects for equality we can write a method 
\mcolor{blue}{\texttt{equals}}.
\end{itemize}

\end{frame}

\begin{frame}
\frametitle{Multiple Constructors}

You may construct objects (as characterized by the field variables) using constructors with different number of arguments (or different types in the arguments).

E.g.,

\verbatiminput{constructor.java}
\end{frame}

\begin{frame}
\frametitle{Problems with User Input}
\mcolor{blue}{How to deal with problems of input?} \\
Not under control of the programmer\\\bigskip
\verbatiminput{no-exception.java}
\end{frame}

\begin{frame}
\frametitle{Exceptions}
Exceptions are used to deal with errors
\verbatiminput{exception.java}
\end{frame}

\begin{frame}
\frametitle{Exceptions (Cont'd)}\vspace{-3ex}\renewcommand{\baselinestretch}{0}\footnotesize
\verbatiminput{exception1.java}
\end{frame}

\begin{frame}
\frametitle{Exceptions in general}
\verbatiminput{exception2.java}
\end{frame}

\begin{frame}
\frametitle{Exceptions in general (Cont'd)}
\verbatiminput{exception3.java}
\end{frame}

\begin{frame}
\frametitle{Exceptions and \texttt{finally}}
\verbatiminput{exception4.java}

Make sure that code in \texttt{catch} and \texttt{finally} never crashes!
\end{frame}

\begin{frame}
\frametitle{Checked vs Unchecked Exceptions}

\begin{itemize}
\item \mcolor{blue}{Unchecked Exceptions} may or may not be caught by the program.\\
  They deal typically with problems that are under control of the
  programmer (e.g., an \texttt{ArrayIndexOutOfBoundsException})
\item \mcolor{blue}{Checked Exceptions} must be caught by the program.
  These deal typically with problems that are NOT under control of the
  programmer (e.g.\ whether a file exists or is accessible). The Java
  compiler enforces a catch statement for a checked exception.
\end{itemize}
\end{frame}

\begin{frame}
\frametitle{Scanner for Input}
\verbatiminput{scanner.java}
\end{frame}

\begin{frame}
\frametitle{Patterns}
\verbatiminput{pattern.java} 
For a full description see 
\href{http://download.oracle.com/javase/1.4.2/docs/api/java/util/regex/Pattern.html}{download.oracle.com/javase/1.4.2/docs/api/java/util/regex/Pattern.html}.

\end{frame}

\begin{frame}
\frametitle{Pattern to Restrict Input for Scanner}
\verbatiminput{scanner-pattern.java}
\end{frame}


\begin{frame}
\frametitle{Reading from/Writing to File}\renewcommand{\baselinestretch}{0.8}\small 
\verbatiminput{io.java}
\end{frame}


\begin{frame}
\frametitle{Reading from a Web page}
\begin{small}
\verbatiminput{html.java}
\end{small}
\end{frame}

\begin{frame}
\frametitle{Throwing Exceptions}
\verbatiminput{throw.java}
\end{frame}

\begin{frame}
  \frametitle{Class Invariants} 
  In classes the implementer may want to enforce that certain field
  variables can take values only in a restricted form, e.g., for a
  variable \mcolor{blue}{\texttt{private String months}} not every value may be
  allowed, but only one of \mcolor{blue}{\texttt{"January"}}, $\ldots$, \mcolor{blue}{\texttt{"December"}}.

Likewise that a variable \mcolor{blue}{\texttt{private String gender}} takes only the values \mcolor{blue}{\texttt{"m"}}, \mcolor{blue}{\texttt{"f"}}, or \mcolor{blue}{\texttt{"x"}}.

 If this is always the case then this is called a \mcolor{blue}{Class Invariant}.

 This can be achieved by throwing an exception whenever with a
 constructor or a setter it is tried to give the variable a value that
 is not allowed.
\end{frame}

\begin{frame}
\frametitle{Assertions}

Assertions are used to establish that properties we are certain that
they hold at a particular point actually do hold. If not an exception
will be raised -- assumed the compiler is correspondingly configured
(by \texttt{-ea} option in `Run Configurations' and `(x)= Arguments'
under `VM Arguments' in Eclipse). Good for debugging.

\verbatiminput{assert.java}
\end{frame}

\end{document}
